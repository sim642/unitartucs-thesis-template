\section{Other Ways to Represent Data}

\subsection{Tables}

\begin{table}[h]
\centering
\caption{Statements in the \proveit language.}
\begin{tabular}{ll}
	\toprule
	Statement & Typeset Example \\
	\midrule
	assignment & $a := 5 + b$ \\
	uniform choice & $m \gets M$ \\
	function signature & $f : K \times M \to L$ \\
	\bottomrule
\end{tabular}
\label{tab:statements}
\end{table}


\subsection{Lists}

Numbered list example:
\begin{enumerate}
	\item item one;
	\item item two;
	\item item three.
\end{enumerate}

\subsection{Math mode}
Example:
\begin{equation}
a + b = c + d
\end{equation}
Aligning:
\begin{align*}
	a &= 5 \\
	b + c &= a \\
	a -2*3 &= 5/4
\end{align*}
Hint: Variables or equations in text are separated with \$ sign, e.g. $a$, $x - y$.

\subsection{Pseudocode}

\begin{figure} [htb]
\begin{lstlisting}
expression
  : NUMBER
  | VARIABLE
  | '+' expression
  | expression '+' expression
  | expression '*' expression
  | function_name '(' parameters ')'
  | '(' expression ')'
\end{lstlisting}
\caption{Grammar of arithmetic expressions.}
\label{fig:parser_exp}
\end{figure}

\subsection{Frame Around Information}

Tip: We can use minipage to create a frame around some important information.
\begin{figure} [h]
\frame{
\begin{minipage}{\textwidth}
\begin{enumerate}
	\item integer division ($\opDiv$) -- only usable between \typeInt types
	\item remainder ($\%$) -- only usable between \typeInt types
\end{enumerate}
\end{minipage}
}
\caption{Arithmetic operations in \proveit revisited.}
\label{fig:aritmOp_revisit}
\end{figure}
