\section{Using References and Citations}
\label{sec:refs-cites}

\subsection{Using References}
LaTeX automatically numbers all kinds of objects, for example:
\begin{itemize}
    \item sections, subsections, subsubsections;
    \item figures, tables, subfigures, subtables;
    \item equations.
\end{itemize}
To be able to reference them, you must add a \emph{label} using \verb|\label{some-label}| just after the numbered object.
Usually, labels for different kinds of objects use corresponding prefixes, for example, \verb|sec:| for sections, \verb|fig:| for figures, \verb|tab:| for tables, \verb|eq:| for equations.

To reference an object by its label, use \verb|\ref{some-label}|.
For example, Section~\ref{sec:refs-cites}, Section~\ref{sec:cites}, Figure~\ref{fig:fnCompModel}, Table~\ref{tab:statements}, Figure~\ref{fig:graph}.
To reference equations, use \verb|\eqref{some-label}| instead, for example Equation~\eqref{eq:abcd}.
There should be a \emph{non-breaking space} \verb|~| between the name of the object and the reference command so that they would not appear on different lines.


\subsection{Using Citations}
\label{sec:cites}

To be able to cite sources (like articles, books, websites) you must add them to the \verb|thesis.bib| file in BibTeX format.
In addition to defining fields like \verb|author| and \verb|title|, each entry also has a unique \emph{key}.
Most publishers (like Springer, ACM) and databases (like DBLP, Google Scholar) offer export for BibTeX entries that can be copied into the bibliography file.
You can manage the bibliography file directly or use a frontend like JabRef\footnote{\url{https://jabref.sourceforge.net/}}.

To cite a source, use \verb|\cite{some-key}| or \verb|\cite[page]{some-key}|, also prefixed with a non-breaking space \verb|~|.
For example, see the following sentences.
Game-based proving is a way to analyse the security of a cryptographic protocol~\cite{GameB_1,GameB_2}.
There are automatic provers, such as CertiCrypt~\cite{certiCrypt} and ProVerif~\cite{proVerif}.
To use a citation as the subject, use \verb|\textcite{some-key}|.
For example, \textcite{certiCrypt} have developed CertiCrypt.

BibLaTeX takes care of the rest automatically: it numbers/names bibliography entries for citations and produces a matching bibliography section with only cited entries.
See TODOs in \verb|thesis.tex| to change the citation and bibliography style (numeric or alphabetic).

% Examples from thesis guidelines.
\nocite{ex5,ex6,ex7,ex8,ex9,ex10,ex11,ex12,chatgpt,copilot}
