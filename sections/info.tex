\newpage
\pdfbookmark[section]{\infoname}{toc} % add info page link to PDF bookmarks, but not table of contents itself

%=== Info in English
\begin{info}
\begin{abstract}
Many interpreting program languages are dynamically typed, such as Visual Basic or Python. As a result, it is easy to write programs that crash due to mismatches of provided and expected data types.  One possible solution to this problem is automatic type derivation during compilation. In this work, we consider study how to detect type errors in the \textsc{Whitespace} language by using fourth-order logic formulae as annotations. The main result of this thesis is a new triple-exponential type inference algorithm for the fourth-order logic formulae. This is a significant advancement as the question of whether there exists such an algorithm was an open question.
All previous attempts to solve the problem lead to logical inconsistencies or require tedious user interaction in terms of interpretative dance. Although the resulting algorithm is slightly inefficient, it can be used to detect obscure programming bugs in the \textsc{Whitespace} language. The latter significantly improves productivity. Our practical experiments showed that productivity is comparable to average Java programmer.
From a theoretical viewpoint, the result is only a small advancement in a rigorous treatment of higher-order logic formulae. The results obtained by us do not generalise to formulae with the fifth or higher order.
\end{abstract}

\keywords{\TODO{List of keywords}}
%Layout, formatting, template

\cercs{\TODO{CERCS code and name:~\url{https://www.etis.ee/Portal/Classifiers/Details/d3717f7b-bec8-4cd9-8ea4-c89cd56ca46e}}}
\end{info}


%=== Info in Estonian
\begin{otherinfo}{estonian}{Tüübituletus neljandat järku loogikavalemitele}
\begin{abstract}
\TODO{One or two sentences providing a basic introduction to the field, comprehensible to a scientist in
any discipline.}
\TODO{Two to three sentences of
more detailed background, comprehensible to scientists in related disciplines.}
\TODO{One sentence clearly stating the general problem being addressed by this particular
study.}
\TODO{One sentence summarising the main result (with the words ``here we show´´ or their equivalent).}
\TODO{Two or three sentences explaining what
the main result reveals in direct
comparison to what was thought to be the case previously, or how the main result adds to previous knowledge.}
\TODO{One or two sentences to put the results into a more general context.}
\TODO{Two or three sentences to provide a
broader perspective, readily
comprehensible to a scientist in any
discipline, may be included in the first paragraph
if the editor considers that the accessibility of
the paper is significantly enhanced by their inclusion.}
\end{abstract}

\keywords{\TODO{List of keywords}}
%Layout, formatting, template

\cercs{\TODO{CERCS kood ja nimetus:~\url{https://www.etis.ee/Portal/Classifiers/Details/d3717f7b-bec8-4cd9-8ea4-c89cd56ca46e}}}
\end{otherinfo}
